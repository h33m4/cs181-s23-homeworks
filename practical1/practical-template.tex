\documentclass[11pt]{article}

\usepackage{common}
\usepackage{booktabs}
\usepackage{hyperref}
\hypersetup{
    colorlinks=true,
    linkcolor=blue,
    filecolor=magenta,      
    urlcolor=cyan,
    pdftitle={Overleaf Example},
    pdfpagemode=FullScreen,
    }
    
\title{CS181 Spring 2023 Practical I: Classifying Sounds}
\author{Adam Mohamed \\ adammohamed@college.harvard.edu}
\date{May 12, 2023}
\begin{document}


\maketitle{}

% Table showing performance of logistic regression baseline models
\begin{table}
\centering
\begin{tabular}{llrrrr}
  &  & Test &   & Train &\\
 \toprule
 Class &  & Raw Ampl. (\%) & Mel Spectogram (\%) & Raw Ampl. (\%) & Mel Spectogram\\
 \midrule
 \textsc{Overall Accuracy} & & 17.89 & 0.60\\
 \textsc{Air Conditioner} & & 28.67 & 5.00  \\
 \textsc{Car Horn} & & 0.00 & 17.54  \\
 \textsc{Children Playing} & & 34.45 & 16.77 \\
 \textsc{Dog Bark} & & 12.23 & 9.82 \\
 \textsc{Drilling} & & 1.14 & 9.82 \\
 \textsc{Engine Idling} & & 30.68 & 9.82 \\
 \textsc{Gun Shot} & & 6.67 & 9.82 \\
 \textsc{Jackhammer} & & 4.24 & 9.82 \\
 \textsc{Siren} & & 13.98 & 9.82 \\
 \textsc{Street Music} & & 15.67 & 9.82 \\
 \bottomrule
\end{tabular}

\caption{\label{tab:results} Baseline models' accuracy for train and test data}
\end{table}


\section{Introduction}
This paper explores the use of linear and nonlinear models in the classification of sounds from the \href{https://urbansounddataset.weebly.com/urbansound8k.html}{UrbanSound8k dataset}.

\section{Part A: Feature Engineering, Baseline Models}

\subsection{Approach}

For the baseline models, I used sklearn's LogisticRegression class to train a linear classification model on both datasets (Raw Amplitude and Mel Spectogram.) The following hyperparameters were selected:

  \begin{itemize}
  \item Representation: How do the two representations differ? What are potential drawbacks or advantages of each?
  
  \item Logistic regression: Describe how the model you trained predicts output probabilities for each class.

  \end{itemize}

\subsection{Results}

This section should report on the following questions: 

\begin{itemize}
\item  What is the \textbf{overall} and \textbf{per-class} classification accuracy of the models that you implemented?
\end{itemize}

\subsection{Discussion}

This section should report on the following questions: 

\begin{itemize}
  \item Why do you hypothesize one feature representation performed better than the other?  
  \item Do you have an intuition for why these baselines might be performing as well/poorly as they are?
  \end{itemize}

\section{Part B: More Modeling}

\subsection{First Step}

\subsubsection{Approach}

What did you do? Credit will be given for:

  \begin{itemize}
  \item Provide mathematical descriptions or pseudocode to help us understand how the models you tried make predictions and are trained.
  \end{itemize}

\subsubsection{Results}
This section should report on the following questions: 

\begin{itemize}
\item  What is the overall and per-class classification accuracy of the models that you implemented?
\end{itemize}


\subsubsection{Discussion}
Compare your results to the logistic regression models in Part A and discuss what your results imply about the task.


\subsection{Hyperparameter Tuning and Validation}

\subsubsection{Approach}
What did you do? Credit will be given for:

  \begin{itemize}
  \item Making tuning and configuration decisions using thoughtful experimentation.  
    How did you perform your hyperparameter search, and what hyperparameters did you search over?
  \end{itemize}

\subsubsection{Results}
Present your results of your hyperparameter search in a way that best reflects how to communicate your conclusions.

\subsubsection{Discussion}

Why do you expect the tuned models to perform better than the baseline models and the model used in First Step? Discuss your validation strategy and your conclusions.

\section{Final Write-up and Reflections}

\subsection{Discussion:} 

Include your paragraph reflections on the key components listed in the instructions.


\section{Optional Exploration, Part C: Explore some more!}
\subsection{Approach}

What did you do? Credit will be given for:
  \begin{itemize}
  \item Diving deeply into all of the model classes and/or pre-processing algorithms that you tried (rather than just trying off-the-shelf tools with default settings).  When relevant, provide mathematical descriptions or pseudocode to help us understand how the models you tried make predictions and are trained. 
  \end{itemize}
  

\subsection{Results}

Describe your results in a way that is appropriate for the experiments that you ran.

\subsection{Discussion}
Credit will be given for:

  \begin{itemize}
  \item Explaining the your reasoning for why you sequentially chose to
    try the approaches you did (i.e. what was it about your initial
    approach that made you try the next change?).  
  \item Explaining the results.  Did the adaptations you tried improve
    the results?  \textbf{Why or why not?}  Did you do additional tests to
    determine if your reasoning was correct?  
  \end{itemize}
 

\end{document}